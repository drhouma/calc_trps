\documentclass[12pt]{article}
\usepackage{ucs}
\usepackage[utf8x]{inputenc} % Включаем поддержку UTF8
\usepackage[russian]{babel}  % Включаем пакет для поддержки русского языка
\title{S21 Calculator 2.0}
\date{}
\author{sreanna}

\begin{document}
  \maketitle
  \section{Introduction}

  Привет пир! Данный мануал поможет тебе в изучении программного инструментала S21 Calculator 2.0.

  S21 Calculator 2.0 предоставляет вам уникальную возможность производить вычисления с точностью до 7 символов после запятой.

  При желании можно вывести график функции, об этом подробнее в пункте Graphs.

  \section{Makefile}
  Для использования Makefile перейдите в папку C7_SmartCalc_v1.0-0/src
  \begin{enumerate} 
    \item Makefile
      \begin{enumerate} 
          \item make install - Устанавливает s21_calculator.exe в папку s21_calculator
          \item make uninstall - Удаляет s21_calculator.exe
          \item make dvi - создает и открывает техдокументацию
          \item make dist - архивирует проект, и помещает его в папку s21_calculator
          \item make tests - тесты на вычисляющую часть калькулятора
          \item make gcov_report - отчет по покрытию тестов tests
      \end{enumerate}
  \end{enumerate}
  \section{Graphs}

  \begin{enumerate}
    \item Введите функцию, которую вы хотите отрисовать
    \item Нажмите кнопку graph
    \item Введите область значений функции в полях xMin и xMax
    \item Нажмите кнопку plot
  \end{enumerate}
  
  \newline
\end{document}
